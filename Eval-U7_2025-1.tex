\documentclass[addpoints,10pt]{exam}
\usepackage[utf8]{inputenc}
\usepackage[T1]{fontenc}
\usepackage[spanish]{babel}
\usepackage{times}
\usepackage{enumerate}
\usepackage{graphicx}
\usepackage{stfloats}
\usepackage{multirow}
\usepackage{array} 
\usepackage{vhistory}
\usepackage{tcolorbox}
\usepackage{fancyvrb}
\usepackage[lighttt]{lmodern}
\usepackage{listings}
\lstset{basicstyle=\small\ttfamily,
	stringstyle=\ttfamily,
	commentstyle=\color{gray45},
	showstringspaces = false,
	keywordstyle=\bfseries}
\usepackage{multicol,caption,capt-of}
%multi-row
\usepackage{multirow}
\usepackage[left=1.5cm,right=1.5cm,top=1.5cm,bottom=1.5cm]{geometry}

\begin{document}
	
	%Encabezados y pie de página
	%----------------------------------------------------------------------------------------
	\pagestyle{headandfoot}
	\runningheadrule
	\firstpageheader{Sistemas Computacionales}{Evaluación Unidad 7}{Mayo 22 de 2025}
	\runningheader{Sistemas Computacionales}
	{Evaluación Unidades 7, Page \thepage\ of \numpages}
	{Mayo 22 de 2025}
%	\firstpagefooter{Escuela de Ingenierías}{UPB}{Fac. Ing. Eléctrica y Electrónica}
	\runningfooter{Escuela de Ingenierías}{UPB}{Ingeniería en Diseño de Entretenimiento Digital}
	
	\begin{center}
	\begin{tabular}{|m{4cm}|m{8cm}|m{5cm}|}
		\hline 
		\multirow{5}{*}{\includegraphics[width=1\linewidth]{Logo-UPB-2022.pdf}}
		&\multicolumn{2}{c|}{\textbf{Universidad Pontificia Bolivariana - Sede  Medellín}}		\\ \cline{2-3}
		&Curso: \textbf{Sistemas Computacionales}		&Duración: 90 minutos				\\ \cline{2-3}
		&Preparada por:\textbf{ Henry Andrade, IEo, Ph.D.}	&Evaluador:							\\ \cline{2-3}
		&Ing. en Diseño de Entretenimiento Digital	&Prueba: Convencional				\\ \cline{2-3}
		&\multicolumn{2}{l|}{Estudiante:				}										\\ 
		 \\ \hline  % Línea extra para cerrar el borde inferior
		
	\end{tabular} 
\end{center}

	%----------------------------------------------------------------------------------------
	%Cuadro con instrucciones
	%----------------------------------------------------------------------------------------		
	\tcbset{colback=red!5!white, colframe=red!75!black, fonttitle=\bfseries}
	\begin{tcolorbox}[title=Nota Importante]
		Lea cuidadosamente cada una de las preguntas antes de contestar. Durante el examen, no se autoriza el acceso a Internet por medio de ningún dispositivo. Cualquier intento de copia o fraude da lugar a un proceso disciplinario.
	\end{tcolorbox}
	%----------------------------------------------------------------------------------------
	\begin{multicols}{2}
		
	\begin{questions}
		
	\boxedpoints
	\pointname{ Puntos}
	\qformat{Pregunta \thequestion \dotfill \thepoints}  
	
	% ------------------------
	% PREGUNTAS DE SELECCIÓN MÚLTIPLE
	% ------------------------
	\question[3]
	¿Cuál de los siguientes \textbf{mejor} describe la función principal de un \texttt{vertex shader} dentro del \emph{render pipeline}?
	
	\begin{checkboxes}
		\choice Calcula la iluminación final de cada fragmento usando normales y texturas  
		\choice Convierte las coordenadas de cada vértice desde el espacio de mundo hasta el espacio de clip  
		\choice Decide si un triángulo es visible comparando su profundidad con el Z-buffer  
		\choice Combina la salida de varios fragmentos para obtener el color definitivo de cada píxel  
	\end{checkboxes}
	
	\question[3]
	El orden correcto de transformaciones que se aplican a un vértice antes de llegar a coordenadas de pantalla es:
	
	\begin{checkboxes}
		\choice View $\rightarrow$ Model $\rightarrow$ Projection $\rightarrow$ Viewport  
		\choice Model $\rightarrow$ View $\rightarrow$ Projection $\rightarrow$ Viewport  
		\choice Projection $\rightarrow$ View $\rightarrow$ Model $\rightarrow$ Viewport  
		\choice Model $\rightarrow$ Projection $\rightarrow$ View $\rightarrow$ Viewport  
	\end{checkboxes}
	
	\question[3]
	El objetivo principal del \textbf{depth buffer} en la GPU es:
	
	\begin{checkboxes}
		\choice Calcular sombras suaves mediante técnicas de \emph{percentage-closer filtering}  
		\choice Almacenar el índice de textura utilizado por cada fragmento  
		\choice Convertir triángulos en fragmentos aplicando interpolación barycéntrica  
		\choice Resolver la visibilidad comparando la distancia de los fragmentos al observador  
	\end{checkboxes}
	
	\question[3]
	Una ventaja clave de la ejecución masivamente paralela de la GPU, mostrada en los videos de la actividad 1, es:
	
	\begin{checkboxes}
		\choice La reducción del ancho de banda requerido entre CPU y RAM del sistema  
		\choice Procesar miles de vértices o fragmentos de manera simultánea con hardware dedicado  
		\choice Evitar la necesidad de compilar los shaders antes de ejecutarlos  
		\choice Permitir que la CPU entre en modo de suspensión profunda mientras la GPU trabaja  
	\end{checkboxes}
	
	% --- Pregunta sobre "uniform" ----------------------------
	\question[3]
	¿Cuál de las siguientes afirmaciones describe \textbf{mejor} a una variable \texttt{uniform} en GLSL?
	
	\begin{checkboxes}
		\choice Es un dato que cambia para cada vértice y se interpola entre fragmentos  
		\choice Es un valor constante para todas las invocaciones del shader durante una misma llamada a \texttt{glDraw*}, establecido por la aplicación desde la CPU  
		\choice Es una variable cuyo valor se calcula dentro del shader y se comparte con la CPU al finalizar el renderizado  
		\choice Es un registro reservado de la GPU que almacena la profundidad (Z) de cada fragmento renderizado  
	\end{checkboxes}

	
	% =========================
	% 2. PREGUNTAS ABIERTAS (6)
	% =========================
	
	\question[5]
	Describe \emph{paso a paso} las fases principales del \textbf{render pipeline} tradicional de OpenGL desde que los vértices son enviados por la CPU hasta que los colores aparecen en el framebuffer.
	
	\question[5]
	Explica la diferencia conceptual y práctica entre aplicar un color fijo a una malla mediante un atributo por vértice y aplicar una textura 2D. Incluye al menos un caso de uso para cada técnica.
	
	\question[5]
	En tus propias palabras, ¿qué es un \textbf{uniform} y cómo se comunica la CPU con el GPU para actualizar su valor en tiempo real? Ilustra tu explicación con un ejemplo de código breve.
	
	\question[5]
	El tutorial \emph{Adding Some Interactivity} modifica posiciones de vértices según la posición del mouse. Propón y explica dos modificaciones diferentes (una en el vertex shader y otra en el fragment shader) que generen comportamientos visuales novedosos.
	

	
	% =========================
	% 3. ORDENAR / COMPLETAR (6)
	% =========================
	
	\question[3]
	\textbf{Jerarquizar.}  
	Ordena las siguientes etapas del pipeline de renderizado en el orden \emph{exacto} en que se ejecutan (1 = primero, 5 = último):
	
	\begin{enumerate}[A. ]
		\item Rasterization  
		\item Vertex Shader  
		\item Fragment Shader  
		\item Blending / Merge in Framebuffer  
		\item Input Assembly  
	\end{enumerate}
	
	\bigskip
	
	
	\question[3]
	\textbf{Completar enunciado.}  
	
	> Un \rule{3cm}{0.15mm} shader opera sobre cada \rule{3cm}{0.15mm} de una primitiva y produce datos que, tras la rasterización, se interpolarán para ser consumidos por la siguiente etapa.
	
	\bigskip
	
	\question[3]
	\textbf{Completar código.}  
	Completa el fragmento GLSL para desplazar cada vértice verticalmente según el tiempo transcurrido:
	
	\begin{verbatim}
		// Vertex Shader (incompleto)
		uniform float u_time;
		
		void main(){
			vec4 pos = vec4(a_position, 1.0);
			// ---- TU LÍNEA AQUÍ ----
			gl_Position = projectionMatrix * modelViewMatrix * pos;
		}
	\end{verbatim}
	
	\bigskip
	
	\question[3]
	\textbf{Jerarquizar.}  
	Ordena de \textbf{más cercano (1) a más lejano (4)} al observador los valores de profundidad (\emph{depth}) que se podrían escribir en el Z-buffer de OpenGL (asumiendo profundidad normalizada [0, 1]):
	
	\begin{enumerate}[A. ]
		\item 0.4  
		\item 0.9  
		\item 0.1  
		\item 0.7  
	\end{enumerate}
	
	\bigskip
	
	\question[3]
	\textbf{Completar enunciado.}  
	
	> El \rule{3cm}{0.15mm} es un área de memoria que almacena el color de cada píxel resultante, mientras que el \rule{3cm}{0.15mm} almacena información de profundidad para resolver la visibilidad de los fragmentos.
		
	\end{questions}
	

	%Sección de preguntas	
	%----------------------------------------------------------------------------------------	

	%----------------------------------------------------------------------------------------
	%Código para insertar imágenes
	%----------------------------------------------------------------------------------------	

%		\begin{figure}[h]
%			\centering
%			\includegraphics[width=0.5\linewidth]{Cto1}
%			\captionof{figure}{Circuito 1}
%			\label{fig:cto1}
%		\end{figure}
	%----------------------------------------------------------------------------------------		
		

	\end{multicols}
	%----------------------------------------------------------------------------------------
	%Tabla con puntajes para evaluación	
	%----------------------------------------------------------------------------------------
	\begin{center}
		\resizebox{\linewidth}{!}{
		\gradetable[h][questions]
	}
	\end{center}
	%----------------------------------------------------------------------------------------	
%	\newpage
	% Start of the revision history table
%	\begin{versionhistory}
%		\vhEntry{1.0}{15.02.17}{HAC}{created}
		%\vhEntry{1.1}{23.01.04}{DP|JPW}{correction}
		%\vhEntry{1.2}{03.02.04}{DP|JPW}{revised after review}
%	\end{versionhistory}
	
\end{document}